%%%%%%%%%%%%%%%%%%%%%%%%%%%%%%%%%%%%%%%%%
% Beamer Presentation
% LaTeX Template
% Version 1.0 (10/11/12)
%
% This template has been downloaded from:
% http://www.LaTeXTemplates.com
%
% License:
% CC BY-NC-SA 3.0 (http://creativecommons.org/licenses/by-nc-sa/3.0/)
%
%%%%%%%%%%%%%%%%%%%%%%%%%%%%%%%%%%%%%%%%%

%----------------------------------------------------------------------------------------
%	PACKAGES AND THEMES
%----------------------------------------------------------------------------------------

\documentclass{beamer}

\mode<presentation> {

\usetheme{Singapore}
\usecolortheme{lily}


\setbeamertemplate{footline}[page number]
\setbeamertemplate{navigation symbols}{}
}

\usepackage{graphicx} % Allows including images
\usepackage{booktabs} % Allows the use of \toprule, \midrule and \bottomrule in tables
\usepackage{polski}
\usepackage[utf8]{inputenc}
\usepackage[ND]{prftree}

% Operators
\newcommand{\imp}{\rightarrow}
\newcommand{\mkcit}[1]{\lbrack{#1}\rbrack}

% Natural Deduction proof-trees
\newcommand{\NDASS}[1]{\prfbyaxiom{Ass}{#1}}

% Linear Logic operators
\newcommand{\LLNEG}[1]{{#1}^\perp}
\newcommand{\LLIMP}{\multimap}
\newcommand{\LLWITH}{\&}
\newcommand{\LLPAR}{\raisebox{\depth}{\scalebox{-1}[-1]{\&}}}
\newcommand{\LLPLUS}{\oplus}
\newcommand{\LLTIMES}{\otimes}

%----------------------------------------------------------------------------------------
%	TITLE PAGE
%----------------------------------------------------------------------------------------

\title[Logiki nieklasyczne]{Logiki nieklasyczne w informatyce}

\author{Szymon Wróbel}

\date{\today}

\begin{document}

\begin{frame}
\titlepage
\end{frame}

\begin{frame}
\frametitle{Plan prezentacji}
\tableofcontents
\end{frame}

%----------------------------------------------
%	PRESENTATION SLIDES
%----------------------------------------------

\section{Wstęp}
\begin{frame}{Po co nam logika?}
  \pause \textbf{Filozofowie:} Poszukiwanie prawdy.

  \pause \textbf{Lingwiści:} Formalizacja znaczenia wypowiedzi.

  \pause \textbf{Matematycy:} Systemy dowodzenia \pause (lub z nudów).

  \pause \textbf{Informatycy:} Weryfikacja poprawności programów.
\end{frame}

\begin{frame}{Powstawanie logiki}
  \begin{columns}
    \begin{column}{0.5\textwidth}
      Projektowanie aplikacji

      \begin{itemize}[<+->]
        \item Problem
        \item Pomysł
        \item Specyfikacja
        \item Implementacja
        \item Wdrożenie
      \end{itemize}
    \end{column}

    \pause

    \begin{column}{0.5\textwidth}
      Projektowanie logiki
      \begin{itemize}[<+->]
        \item Problem
        \item Intuicja
        \item Składnia
        \item Semantyka
        \item Zastosowania
      \end{itemize}
    \end{column}
  \end{columns}
\end{frame}

\section{Logika intuicjonistyczna}
\begin{frame}{Dowody konstruktywne}
  \begin{block}{Problem}
  Czy istnieją dwie liczby niewymierne $a, b$, takie, że $a^b$ jest liczbą wymierną?
  \end{block}
  
  \pause
  
  \begin{block}{Dowód}
  Weźmy $a=\sqrt{2}, b=\sqrt{2}$. Rozważmy wymierność $a^{b}$. Jeśli jest wymierne, to dowód jest zakończony. Jeśli nie, weźmy $a=\sqrt{2}^{\sqrt{2}}, b=\sqrt{2}$.
  
  Wtedy
  
  $$ a^b = {\left( \sqrt{2}^{\sqrt{2}} \right)}^{\sqrt{2}}
  = \sqrt{2}^{\sqrt{2} \ast \sqrt{2}} = \sqrt{2}^2 = 2
  $$
  
  $\blacksquare$
  
  \end{block}
\end{frame}

\begin{frame}{Dowody konstruktywne}
  \begin{block}{Problem}
  Podaj dwie liczby niewymierne $a, b$, takie, że $a^b$ jest liczbą wymierną?
  \end{block}
  
  \pause
  
  \begin{alertblock}{}
  Pomimo tego, że udowodniliśmy istnienie tych liczb, nie możemy skorzystać z poprzedniego dowodu
  \end{alertblock}
\end{frame}

\begin{frame}{Dowody konstruktywne}
  \begin{block}{Dowód (v 2.0)}
  Weźmy $ a=\sqrt{2}, b=2\log_2{3} $.
  
  Wtedy $ a^b = \sqrt{2}^{(2\log_2{3})} = 2^{\log_2{3}} = 3 $
  
  $\blacksquare$
  \end{block}
\end{frame}

\begin{frame}{Logika intuicjonistyczna}
    \only<1>{
      \begin{exampleblock}{}
      Prawdziwe jest to, na co mamy dowód.
      \end{exampleblock}
    }

    \pause
    
    \begin{block}{Interpretacja BHK}
        \begin{itemize}[<+->]
            \item Dowód $ A \land B $ to dowód A i dowód B
            \item Dowód $ A \lor B $ to dowód A albo dowód B
            \item Dowód $ A \imp B $ to metoda przekształcająca dowód A, w dowód B
            \item Nie ma dowodu $\bot$
        \end{itemize}
    \end{block}
\end{frame}

\begin{frame}{Dedukcja naturalna}
  \only<1>{
    \begin{block}{Postać sekwentów}
      $$ \Delta \vdash \Gamma $$
    \end{block}
  }

  \pause

  \begin{block}{Aksjomat}
    $$ \NDASS{\Delta, P \vdash P} $$
  \end{block}

  \pause

  \begin{block}{Implikacja}
    \begin{columns}
      \begin{column}{0.5\textwidth}
        $$
        \NDIMPI {\Delta, A \vdash B}
          {\Delta \vdash A \imp B}
        $$
      \end{column}

      \begin{column}{0.5\textwidth}
        $$
        \NDIMPE {\Delta \vdash A \imp B} {\Delta \vdash A}
          {\Delta \vdash B}
        $$
      \end{column}
    \end{columns}
  \end{block}

  \pause

  \begin{block}{Koniunkcja}
    \begin{columns}
      \begin{column}{0.33\textwidth}
        $$
        \NDANDI {\Delta \vdash A} {\Delta \vdash B} 
          {\Delta \vdash A \land B}
        $$
      \end{column}

      \begin{column}{0.33\textwidth}
        $$
        \NDANDEL {\Delta \vdash A \land B}
          {\Delta \vdash A}
        $$
      \end{column}

      \begin{column}{0.33\textwidth}
        $$
        \NDANDER {\Delta \vdash A \land B}
          {\Delta \vdash B}
        $$
      \end{column}
    \end{columns}
  \end{block}
\end{frame}

\begin{frame}{Dedukcja naturalna}
  \begin{block}{Alternatywa}
    \begin{columns}
      \begin{column}{0.5\textwidth}
        $$
        \NDORIL
          {\Delta \vdash A}
          {\Delta \vdash A \lor B}
        $$
      \end{column}

      \begin{column}{0.5\textwidth}
        $$
        \NDORIR
          {\Delta \vdash B}
          {\Delta \vdash A \lor B}
        $$
      \end{column}
    \end{columns}

    \begin{columns}
      \begin{column}{1\textwidth}
        $$
        \NDORE {\Delta \vdash A \lor B} {\Delta, A \vdash C} {\Delta, B \vdash C}
          {\Delta \vdash C}
        $$
      \end{column}

    \end{columns}
  \end{block}

  \pause

  \begin{alertblock}{$NI \Rightarrow NK$}
    \begin{columns}
      \begin{column}{0.3\textwidth}
        $$
        \NDP[r]{LEM}
          {}
          {\Delta \vdash P \lor \neg P}
        $$
      \end{column}

      \begin{column}{0.3\textwidth}
        $$
        \NDP[r]{DNE}
          {\Delta \vdash \neg \neg P}
          {\Delta \vdash P}
        $$
      \end{column}

      \begin{column}{0.3\textwidth}
        $$
        \NDP[r]{PBC}
          {\Delta, \neg P \vdash \bot }
          {\Delta \vdash P}
        $$
      \end{column}
    \end{columns}
  \end{alertblock}
\end{frame}

\begin{frame}{Przykład: $P \imp \neg \neg P$}
  \only<1>{
    $$
    { \vdash P \imp (P \imp \bot) \imp \bot }
    $$
  }

  \only<2>{
    $$
    \NDIMPI {
      { P \vdash (P \imp \bot) \imp \bot }
    }
    { \vdash P \imp (P \imp \bot) \imp \bot }
    $$
  }

  \only<3>{
    $$
    \NDIMPI {
      \NDIMPI {
          { P, (P \imp \bot) \vdash \bot }
      } 
      { P \vdash (P \imp \bot) \imp \bot }
    }
    { \vdash P \imp (P \imp \bot) \imp \bot }
    $$
  }

  \only<4>{
    $$
    \NDIMPI {
      \NDIMPI {
        \NDIMPE 
          { {P, (P \imp \bot) \vdash (P \imp \bot)} }
          { {P, (P \imp \bot) \vdash P} }
          { P, (P \imp \bot) \vdash \bot }
      } 
      { P \vdash (P \imp \bot) \imp \bot }
    }
    { \vdash P \imp (P \imp \bot) \imp \bot }
    $$
  }

  \only<5>{
    $$
    \NDIMPI {
      \NDIMPI {
        \NDIMPE 
          { \NDASS {P, (P \imp \bot) \vdash (P \imp \bot)} }
          { \NDASS {P, (P \imp \bot) \vdash P} }
          { P, (P \imp \bot) \vdash \bot }
      } 
      { P \vdash (P \imp \bot) \imp \bot }
    }
    { \vdash P \imp (P \imp \bot) \imp \bot }
    $$
  }
\end{frame}

\section{Inne logiki nieklasyczne}
\begin{frame}{Logika liniowa}
  \begin{block}{Przykład}
  \pause Niech $P$ oznacza "mieć ciastko".

  \pause Niech $Q$ oznacza "zjeść ciastko".

  \pause Jeśli mamy ciastko, to możemy je zjeść, co zapiszemy jako $P \imp Q$

  \pause Wtedy w logice intuicjonistycznej możemy udowodnić $P \imp (P \imp Q) \imp P \land Q $.
  \end{block}

  \begin{alertblock}{Wniosek}
    Dzięki logice intuicjonistycznej możemy zjeść ciastko i mieć ciastko
  \end{alertblock}
\end{frame}

\begin{frame}{Logika liniowa: składnia}
  \begin{block}{Zmienne zdaniowe}
    $$ p, \LLNEG{p} $$
  \end{block}

  \begin{block}{Stałe}
    $$ 1, \bot, \top, 0 $$
  \end{block}

  \begin{block}{Koniunkcja}
    $$ A \LLTIMES B, A \LLWITH B $$
  \end{block}

  \begin{block}{Alternatywa}
    $$ A \LLPAR B, A \LLPLUS B $$
  \end{block}
\end{frame}

\begin{frame}{Logika liniowa: zastosowania}
  \begin{itemize}[<+->]
    \item Typy Liniowe jako kontrakt \mkcit{Wad91}
    \item Logika liniowa jako logika współbieżności/równoległości \mkcit{Wad14}
    \item Powiązania z układami kwantowymi \mkcit{Lag12}, \mkcit{Bae09}
  \end{itemize}

\end{frame}

\begin{frame}{Relevance Logic}

\end{frame}

\begin{frame}{Paraconsistent Logic}

\end{frame}

\begin{frame}{Computability Logic (CoL)}

\end{frame}

\section*{Bibliografia}
\begin{frame}{Bibliografia}
    \begin{itemize}
      \item \mkcit{Tho91} S. Thompson, \textit{Type Theory and Functional Programming}, Addison-Wesley, 1991.
      \item \mkcit{Pri08} G. Priest, \textit{An Introduction to Non-Classical Logic: From If to Is}, Cambridge University Press, 2008.
      \item \mkcit{Gen35} G. Gentzen, \textit{Untersuchungen über das logische Schließen}, 1935.
      \item \mkcit{Wad91} P. Wadler, \textit{Is there a use for linear logic?}, 1991.
      \item \mkcit{Wad14} P. Wadler, \textit{Propositions as sessions}, Journal of Functional Programming, vol. 24,  2014.
      \item \mkcit{Lag12} U. Dal Lago, C. Faggian,  \textit{On Multiplicative Linear Logic, Modality and Quantum Circuits}, Electronic Proceedings in Theoretical Computer Science, 2012
      \item \mkcit{Bae09} J. Baez, M. Stay, \textit{Physics, Topology, Logic and Computation: A Rosetta Stone}, 2009.
    \end{itemize}
\end{frame}


\end{document}

