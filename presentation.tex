%%%%%%%%%%%%%%%%%%%%%%%%%%%%%%%%%%%%%%%%%
% Beamer Presentation
% LaTeX Template
% Version 1.0 (10/11/12)
%
% This template has been downloaded from:
% http://www.LaTeXTemplates.com
%
% License:
% CC BY-NC-SA 3.0 (http://creativecommons.org/licenses/by-nc-sa/3.0/)
%
%%%%%%%%%%%%%%%%%%%%%%%%%%%%%%%%%%%%%%%%%

%----------------------------------------------------------------------------------------
%	PACKAGES AND THEMES
%----------------------------------------------------------------------------------------

\documentclass{beamer}

\mode<presentation> {

\usetheme{Singapore}
\usecolortheme{lily}


\setbeamertemplate{footline}[page number]
\setbeamertemplate{navigation symbols}{}
}

\usepackage{graphicx} % Allows including images
\usepackage{booktabs} % Allows the use of \toprule, \midrule and \bottomrule in tables
\usepackage{polski}
\usepackage[utf8]{inputenc}
%----------------------------------------------------------------------------------------
%	TITLE PAGE
%----------------------------------------------------------------------------------------

\title[Logiki nieklasyczne]{Logiki nieklasyczne w informatyce}

\author{Szymon Wrobel}

\date{\today}

\begin{document}

\begin{frame}
\titlepage
\end{frame}

% \begin{frame}
% \frametitle{Plan prezentacji}
% \tableofcontents
% \end{frame}

%----------------------------------------------
%	PRESENTATION SLIDES
%----------------------------------------------

\section{Wstęp}
\begin{frame}{Rys historyczny}
    \begin{block}{Rachunek sekwentów}
    $$ \Delta \vdash \phi $$
    \end{block}
\end{frame}

\section{Logika intuicjonistyczna}
\begin{frame}{Dowody konstruktywne}
    \begin{block}{Problem}
    Czy istnieją dwie liczby niewymierne $a, b$, takie, że $a^b$ jest liczbą wymierną?
    \end{block}
    
    \pause
    
    \begin{block}{Dowód}
    Weźmy $a=\sqrt{2}, b=\sqrt{2}$. Rozważmy wymierność $a^{b}$. Jeśli jest wymierne, to dowód jest zakończony. Jeśli nie, weźmy $a=\sqrt{2}^{\sqrt{2}}, b=\sqrt{2}$.
    
    Wtedy
    
    $$ a^b = {\left( \sqrt{2}^{\sqrt{2}} \right)}^{\sqrt{2}}
    = \sqrt{2}^{\sqrt{2} \ast \sqrt{2}} = \sqrt{2}^2 = 2
    $$
    
    $\blacksquare$
    
    \end{block}
\end{frame}

\begin{frame}{Dowody konstruktywne}
    \begin{block}{Problem}
    Podaj dwie liczby niewymierne $a, b$, takie, że $a^b$ jest liczbą wymierną?
    \end{block}
    
    \pause
    
    \begin{alertblock}{Uwaga}
    Pomimo tego, że udowodniliśmy istnienie tych liczb, nie możemy skorzystać z poprzedniego dowodu
    \end{alertblock}
\end{frame}

\begin{frame}{Dowody konstruktywne}
    \begin{block}{Dowód (v 2.0)}
    Weźmy $ a=\sqrt{2}, b=2\log_2{3} $.
    
    Wtedy $ a^b = \sqrt{2}^{(2\log_2{3})} = 2^{\log_2{3}} = 3 $
    
    $\blacksquare$
    \end{block}
\end{frame}

\begin{frame}{Logika intuicjonistyczna}

    \begin{exampleblock}{}
    Prawdą jest to, na co jesteśmy w stanie przedstawić dowód.
    \end{exampleblock}
    
    \pause
    
    \begin{block}{Interpretacja BHK}
        \begin{itemize}
            \item Dowód $ A \land B $ to dowód A i dowód B
            \item Dowód $ A \lor B $ to dowód A albo dowód B
            \item Dowód $ A \supset B $ to metoda przekształcająca dowód A, w dowód B
            \item Nie ma dowodu $\bot$
        \end{itemize}
    \end{block}
\end{frame}

\section{Logika liniowa}
\begin{frame}{Title}
    
\end{frame}

\section{Co dalej?}
\begin{frame}{Title}
    
\end{frame}

\section*{Bibliografia}
\begin{frame}{Title}
    
\end{frame}


\end{document}

