%%%%%%%%%%%%%%%%%%%%%%%%%%%%%%%%%%%%%%%%%
% Beamer Presentation
% LaTeX Template
% Version 1.0 (10/11/12)
%
% This template has been downloaded from:
% http://www.LaTeXTemplates.com
%
% License:
% CC BY-NC-SA 3.0 (http://creativecommons.org/licenses/by-nc-sa/3.0/)
%
%%%%%%%%%%%%%%%%%%%%%%%%%%%%%%%%%%%%%%%%%

%----------------------------------------------------------------------------------------
%	PACKAGES AND THEMES
%----------------------------------------------------------------------------------------

\documentclass{beamer}

\mode<presentation> {

\usetheme{Singapore}
\usecolortheme{lily}


\setbeamertemplate{footline}[page number]
\setbeamertemplate{navigation symbols}{}
}

\usepackage{graphicx} % Allows including images
\usepackage{booktabs} % Allows the use of \toprule, \midrule and \bottomrule in tables
\usepackage{polski}
\usepackage[utf8]{inputenc}
\usepackage[ND]{prftree}

\newcommand{\imp}{\rightarrow}
\newcommand{\NDASS}[1]{\prfbyaxiom{Ass}{#1}}

%----------------------------------------------------------------------------------------
%	TITLE PAGE
%----------------------------------------------------------------------------------------

\title[Logiki nieklasyczne]{Logiki nieklasyczne w informatyce}

\author{Szymon Wróbel}

\date{\today}

\begin{document}

\begin{frame}
\titlepage
\end{frame}

\begin{frame}
\frametitle{Plan prezentacji}
\tableofcontents
\end{frame}

%----------------------------------------------
%	PRESENTATION SLIDES
%----------------------------------------------

\section{Wstęp}
\begin{frame}{Wstęp}
  \begin{columns}
    \begin{column}{0.5\textwidth}
      Projektowanie aplikacji

      \begin{itemize}[<+->]
        \item Problem
        \item Pomysł
        \item Specyfikacja
        \item Implementacja
        \item Wdrożenie
      \end{itemize}
    \end{column}

    \pause

    \begin{column}{0.5\textwidth}
      Projektowanie logiki
      \begin{itemize}[<+->]
        \item Problem
        \item Intuicja
        \item Składnia
        \item Semantyka
        \item Systemy dowodzenia
      \end{itemize}
    \end{column}
  \end{columns}
\end{frame}

\begin{frame}{Dedukcja naturalna $NK$}
  \only<1>{
    \begin{block}{Postać sekwentów}
      $$ \Delta \vdash \Gamma $$
    \end{block}
  }

  \pause

  \begin{block}{Aksjomat}
    $$ \NDASS{\Delta, P \vdash P} $$
  \end{block}

  \pause

  \begin{block}{Implikacja}
    \begin{columns}
      \begin{column}{0.5\textwidth}
        $$
        \NDIMPI {\Delta, A \vdash B}
          {\Delta \vdash A \imp B}
        $$
      \end{column}

      \begin{column}{0.5\textwidth}
        $$
        \NDIMPE {\Delta \vdash A \imp B} {\Delta \vdash A}
          {\Delta \vdash B}
        $$
      \end{column}
    \end{columns}
  \end{block}

  \pause

  \begin{block}{Koniunkcja}
    \begin{columns}
      \begin{column}{0.33\textwidth}
        $$
        \NDANDI {\Delta \vdash A} {\Delta \vdash B} 
          {\Delta \vdash A \land B}
        $$
      \end{column}

      \begin{column}{0.33\textwidth}
        $$
        \NDANDEL {\Delta \vdash A \land B}
          {\Delta \vdash A}
        $$
      \end{column}

      \begin{column}{0.33\textwidth}
        $$
        \NDANDER {\Delta \vdash A \land B}
          {\Delta \vdash B}
        $$
      \end{column}
    \end{columns}
  \end{block}
\end{frame}

\begin{frame}{Dedukcja naturalna $NK$}
  \begin{block}{Alternatywa}
    \begin{columns}
      \begin{column}{0.5\textwidth}
        $$
        \NDORIL
          {\Delta \vdash A}
          {\Delta \vdash A \lor B}
        $$
      \end{column}

      \begin{column}{0.5\textwidth}
        $$
        \NDORIR
          {\Delta \vdash B}
          {\Delta \vdash A \lor B}
        $$
      \end{column}
    \end{columns}

    \begin{columns}
      \begin{column}{1\textwidth}
        $$
        \NDORE {\Delta \vdash A \lor B} {\Delta, A \vdash C} {\Delta, B \vdash C}
          {\Delta \vdash C}
        $$
      \end{column}

    \end{columns}
  \end{block}

  \pause

  \begin{alertblock}{LEM}
    \begin{columns}
      \begin{column}{1\textwidth}
        $$
        \NDP[r]{LEM}
          {}
          {\Delta \vdash P \lor \neg P}
        $$
      \end{column}
    \end{columns}
  \end{alertblock}
\end{frame}

\section{Logika intuicjonistyczna}
\begin{frame}{Dowody konstruktywne}
  \begin{block}{Problem}
  Czy istnieją dwie liczby niewymierne $a, b$, takie, że $a^b$ jest liczbą wymierną?
  \end{block}
  
  \pause
  
  \begin{block}{Dowód}
  Weźmy $a=\sqrt{2}, b=\sqrt{2}$. Rozważmy wymierność $a^{b}$. Jeśli jest wymierne, to dowód jest zakończony. Jeśli nie, weźmy $a=\sqrt{2}^{\sqrt{2}}, b=\sqrt{2}$.
  
  Wtedy
  
  $$ a^b = {\left( \sqrt{2}^{\sqrt{2}} \right)}^{\sqrt{2}}
  = \sqrt{2}^{\sqrt{2} \ast \sqrt{2}} = \sqrt{2}^2 = 2
  $$
  
  $\blacksquare$
  
  \end{block}
\end{frame}

\begin{frame}{Dowody konstruktywne}
  \begin{block}{Problem}
  Podaj dwie liczby niewymierne $a, b$, takie, że $a^b$ jest liczbą wymierną?
  \end{block}
  
  \pause
  
  \begin{alertblock}{}
  Pomimo tego, że udowodniliśmy istnienie tych liczb, nie możemy skorzystać z poprzedniego dowodu
  \end{alertblock}
\end{frame}

\begin{frame}{Dowody konstruktywne}
  \begin{block}{Dowód (v 2.0)}
  Weźmy $ a=\sqrt{2}, b=2\log_2{3} $.
  
  Wtedy $ a^b = \sqrt{2}^{(2\log_2{3})} = 2^{\log_2{3}} = 3 $
  
  $\blacksquare$
  \end{block}
\end{frame}

\begin{frame}{Logika intuicjonistyczna}
    \only<1>{
      \begin{exampleblock}{}
      Prawdziwe jest to, na co mamy dowód.
      \end{exampleblock}
    }

    \pause
    
    \begin{block}{Interpretacja BHK}
        \begin{itemize}[<+->]
            \item Dowód $ A \land B $ to dowód A i dowód B
            \item Dowód $ A \lor B $ to dowód A albo dowód B
            \item Dowód $ A \supset B $ to metoda przekształcająca dowód A, w dowód B
            \item Nie ma dowodu $\bot$
        \end{itemize}
    \end{block}
\end{frame}

\begin{frame}{Dedukcja naturalna $NI$}
  \only<1> {
    \begin{block}{Aksjomat}
      $$ \NDASS{\Delta, P \vdash P} $$
    \end{block}
  }

  \pause

  \begin{block}{Implikacja}
    \begin{columns}
      \begin{column}{0.5\textwidth}
        $$
        \NDIMPI {\Delta, A \vdash B}
          {\Delta \vdash A \imp B}
        $$
      \end{column}

      \begin{column}{0.5\textwidth}
        $$
        \NDIMPE {\Delta \vdash A \imp B} {\Delta \vdash A}
          {\Delta \vdash B}
        $$
      \end{column}
    \end{columns}
  \end{block}

  \pause

  \begin{block}{Koniunkcja}
    \begin{columns}
      \begin{column}{0.33\textwidth}
        $$
        \NDANDI {\Delta \vdash A} {\Delta \vdash B} 
          {\Delta \vdash A \land B}
        $$
      \end{column}

      \begin{column}{0.33\textwidth}
        $$
        \NDANDEL {\Delta \vdash A \land B}
          {\Delta \vdash A}
        $$
      \end{column}

      \begin{column}{0.33\textwidth}
        $$
        \NDANDER {\Delta \vdash A \land B}
          {\Delta \vdash B}
        $$
      \end{column}
    \end{columns}
  \end{block}

  \pause

  \begin{block}{Alternatywa}
    \begin{columns}
      \begin{column}{0.5\textwidth}
        $$
        \NDORIL
          {\Delta \vdash A}
          {\Delta \vdash A \lor B}
        $$
      \end{column}

      \begin{column}{0.5\textwidth}
        $$
        \NDORIR
          {\Delta \vdash B}
          {\Delta \vdash A \lor B}
        $$
      \end{column}
    \end{columns}

    \begin{columns}
      \begin{column}{1\textwidth}
        $$
        \NDORE {\Delta \vdash A \lor B} {\Delta, A \vdash C} {\Delta, B \vdash C}
          {\Delta \vdash C}
        $$
      \end{column}

    \end{columns}
  \end{block}

\end{frame}

\begin{frame}{Przykład: $P \imp \neg \neg P$}
  $$
    \NDIMPI {
      \NDIMPI {
        \NDIMPE 
          { \NDASS {P, (P \imp \bot) \vdash (P \imp \bot)} }
          { \NDASS {P, (P \imp \bot) \vdash P} }
          { P, (P \imp \bot) \vdash \bot }
      } 
      { P \vdash (P \imp \bot) \imp \bot }
    }
    { \vdash P \imp (P \imp \bot) \imp \bot }
  $$
\end{frame}

\section{Logika liniowa}
\begin{frame}{Title}
    
\end{frame}

\section{Inne logiki}
\begin{frame}{Title}
    
\end{frame}

\section*{Bibliografia}
\begin{frame}{Bibliografia}
    \begin{itemize}
      \item S. Thompson, Type Theory and Functional Programming, Addison-Wesley, 1991.
      \item G. Priest, An Introduction to Non-Classical Logic: From If to Is, 2nd ed. Cambridge: Cambridge University Press, 2008.
      \item G. Gentzen, Untersuchungen über das logische Schließen I, 1935.
    \end{itemize}
\end{frame}


\end{document}

